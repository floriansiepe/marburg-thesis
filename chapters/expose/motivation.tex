\section{Motivation}
\label{sec:motivation}

The motivation should introduce your research question and motivate your novel solution attempt: What is the problem? Why do we care? What is the challenge? Why has nobody else done or solved this?

If the thesis is about some specific use case or domain problem, introduce the reader to that domain. Sometimes it is very helpful to provide examples or illustrations for the introductions, such as a motivating experimental result, a depiction of the domain, or an introduction to an interesting project partner.

Always provide sources (in form of references or cites) for statements that are made in the introduction. \emph{Competitors are slow} - state where this was measured! \emph{Data volumes are increasing steadily} - state who investigated this! \emph{Data scientists require a solution for X} - state which sources makes this clear! ...

This is an Exposé, so there are no definite results to be advertised, but we can highlight the potential of a new idea, best- and worst-case result expectations, promising first experimental results, intuitions for success, and/or the necessity for certain developments.

Specify your research questions: The (2-4) research questions will later become the contributions of your Thesis. At this point, the questions should be interesting and challenging questions that cannot be answered yet - otherwise, they are probably no research questions. Questions usually have the form: How can we apply X to Z? How can we parallelize/distribute/approximate X? Can X be learned/quantified? What is the effect of X when we do Y? ...

Provide a short overview of this Exposé at the end.

Rename the section to something else than ``Motivation''. Sections should always have meaningful names and not the names of their function (well, ``Related Work'' and ``Evaluation'' are usually the exception, but all other section need to be renamed!). For example: ``Motivation'' could be a general, Domain-specific label, such as ``Multivariate Time Series Analytics'' or ``Relational Data Profiling'', and ``Approach'' could be the novel, to-be-developed system, such as ``Holistic Anomaly Detection'' or ``Actor-based IND Discovery''.
